\documentclass{NSF}

%\graphicspath{{figures/}}

%\fontfamily{ppl}\selectfont
%\setmainfont{Arial}
%\usepackage{fontspec}
%\usepackage{palatino}
\fontfamily{cmss}\selectfont
\usepackage[numbib]{tocbibind}
\usepackage[T1]{fontenc}
\usepackage{cmbright}
\usepackage{color}
\usepackage{comment}
\usepackage{pdfpages}

\newcommand{\bo}[1]{\textcolor{blue}{Bo: #1}}
\newcommand{\Xu}[1]{\textcolor{blue}{Xu: #1}}

\begin{document}

\setcounter{section}{1}

\noindent \textbf{Thrust AI-2: Multi-Agent Collaboration. (Xiaoming Huo)}

\textbf{This section should be around 2 pages long; However, we apparently exceed this limit. I choose not to cut it at this stage.}


A domain task may involve multiple agents, either within or across organizational boundaries (e.g., when detecting ``cyber-pandemics”).
We propose to develop a multi-agent collaboration framework, enabled by research in the following aspects.
\begin{enumerate}
\item Formal modeling of multi-agent networks:
We will develop new concepts and models to assess the aggregate intelligence and capability of a network of agents performing a domain task.
The models will capture both per-agent characteristics (e.g., knowledge and capability) as well as inter-agent relationships, such as trust and degree of cooperation.
New theories will be developed to quantify the network-wide intelligence and capability, based on derived properties such as scale, agreement/polarization~\cite{wang22coevolution}, and contagion (e.g., of misinformation) status.

\item
Distributed learning and reasoning:
We will study the effectiveness of multi-agent learning and reasoning under a variety of factors, such as the type/amount of knowledge shared, adversarial disturbances~\cite{SilvestreRosaHespanhaSilvestreJun14}, and choice of reward functions and action quality assessment methods~\cite{radha_arel}.

\item Reinforcement Learning:
%\textcolor{red}{xxx- Add some overview statement here}
We will apply methodology from reinforcement learning in the multi-agent setting.
We will consider the topics such as integrating the deep learning technique, utilizing theory developed for unsupervised reinforcement learning, and their adaptation into the multi-agent framework.

\item
Multi-agent network as moving target:
To mitigate adversarial disturbances, we will turn a multi-agent network into an intelligent moving target by dynamically changing its organization/configuration, driven by domain knowledge and real-time situation-awareness, to avoid sophisticated adversarial analysis, evasion, and influence.
\textcolor{red}{xxx- Need more contents on this topic...}


\end{enumerate}



\noindent \textbf{1. Formal Modeling of Multi-agent Networks}

We describe the modeling aspect of the multi-agent network system.

\noindent {\bf Network Models.}
We consider a distributed collection of AI agents for monitoring and detecting security threats.
The model will include node features including agent-level characteristics such as agent-specific knowledge, specialties (sensing, actuation, and decision making capabilities), and tasks.
Some of the node features are probabilistic in nature, for example, the chance or belief of certain security attacks or abnormal behaviors.
We will include features on edges or hyper-edges capturing properties of interactions and relationships among agents such as trust, information exchange or aggregation patterns.

\noindent {\bf Information Exchange and Contagion.}
Both node features and edge features are dynamically evolving.
This will enable the distributed agents to share their individual discoveries such that a `pandemic' attack can be detected quickly, and agents that have not yet experienced attacks can be well prepared.
We propose co-evolving models where node features especially the knowledge of current vulnerabilities and potential attacks is exchanged.
The update pattern and frequency take considerations such as whether two agents belong to the same organization and whether their tasks are compatible/overlapping.

\noindent {\bf Consensus and Polarization.}
Our goal is to enable agents sharing their knowledge and new discoveries with other agents in a network using a proper contagion model such that all agents in the network can provide timely and intelligent responses to emerging security concerns.
Towards this objective, we will develop new theory on understanding the global network behaviors arising from local interactions among autonomous agents.
We will develop a suite of tools and protocols for how and when agents communicate with one another with their new discoveries/knowledge models, such that new threats that are independently detected and confirmed by multiple agents in the network can be quickly recognized and acted upon in a timely manner.
We plan to derive conditions on network-wide convergence (consensus) or partition (polarization), with dependency on network topology and interaction/contagion models.
The work builds upon prior literature on contagion models and network consensus~\cite{Ghasemiesfeh:2013:CCW,gao2017engineering,ebrahimi17complex,gao19volatility,gao16general,wang22coevolution}, but expects to embrace much more complex agent behaviors and interaction patterns.
In particular, we need to balance between a number of considerations:
1) fast convergence and spreading of new intelligence through the network;
2) having the capability of controlling or pruning misinformation;
3) embracing locality where nodes closer in the network should share more similar views.

Classical theory on contagion models show that simple gossip models (e.g., new information is shared with all neighbors) can quickly spread information to the entire network but misinformation through any single node can also quickly spread to the entire network.
Alternatively, we can introduce more conservative contagion models.
For example, adoption requires separate confirmations from neighbors -- e.g., in $k$-complex contagion model~\cite{Ghasemiesfeh:2013:CCW,ebrahimi17complex,gao16general}, a node adopts the new information only when at least $k$ neighbors have adopted.
We can also take a variant of the `naming game' model~\cite{gao2017engineering} (which explains the emergence of shared conventions), to allow the emergence of suspected attacks that are most persistent in the network---each node keeps a list of their current security profiles; at each round a node takes one in the list and exchanges with a neighbor; security profiles that are confirmed by both nodes will be given higher belief levels in the later rounds.
We will also consider co-evolving models, where the edge features are updated dynamically as well.
For example, if two nodes consistently show differences in their security profiles and tasks, the edge weight should be lowered such that information exchange is less frequent.
In our recent work~\cite{wang22coevolution} we have studied the network co-evolving behavior where network will converge to either a homogeneous consensus one or a polarized one.
We would need a more refined analysis that tie the initial states, the contagion models and the final converged status.
In summary this theory will provide a rigorous foundation for quantifying the collective intelligence and capability of the multi-agent network.


\noindent \textbf{2. Distributed learning and reasoning}

We will study the effectiveness of multi-agent learning and reasoning under a variety of factors, such as the type/amount of knowledge shared, adversarial disturbances, and choice of reward functions and action quality assessment methods.

Our team (Poovendran) has developed solutions to mitigate the impact of sparse rewards in reinforcement learning. 
We designed algorithms to improve credit assignment, i.e., quantifying how good an action taken by the agent is, inspired by physical laws like conservation of energy (when information about the environment was available) ~\cite{xiao2022shaping, xiao2019potential}, using feedback from human operators (when rewards were delayed and difficult to interpret) ~\cite{xiao2020fresh} and self-attention (when rewards were episodic in unknown environments) ~\cite{xiao2022agent}.  
We will translate insights gleaned from this research to reasoning about interactions among multiple heterogeneous agents.

\noindent {\bf Redistribution of Sparse Rewards}:
In many cyber systems, the effects of the actions of an adversary become evident to a user or system administrator only after the adversary has spent a long duration of time in the system~\cite{brogi2016terminaptor, ahmad2019strategically, alshamrani2019survey, huang2019adaptive}. 
An example is an advanced persistent threat (APT), where an adversary is embedded within the host system over long time frames, with an objective to exfiltrate sensitive information, or damage critical infrastructure. 
Since an APT operates stealthily, rewards or penalties are revealed only at an advanced stage of the attack process~\cite{zhu2018multi, moothedath2020game, sahabandu2020quickest}. 
We will develop solutions to effectively redistribute the reward over the duration of the system-adversary interaction so that the system will be able to respond faster and more efficiently to mitigate the impact of the adversary. 
When there is more than one adversary, we will design methods to assess the relative contributions of each adversary to a common global reward. 
Such an assessment will contribute to the design of scheduling protocols to determine the priority with which each adversarial threat must be examined. 
We will investigate adversaries with different scopes---ranging from identical (fully homogeneous) to distinct (fully heterogeneous)---and design efficient redistribution algorithms when rewards are delayed.

\noindent {\bf Distracting and Deceptive Rewards}:
Current reinforcement learning paradigms presume that rewards received by an agent have high fidelity. 
Shortcomings in the specification of a reward have typically been addressed through `reward hacking’ mechanisms that resort to optimizing a proxy reward that is more easily measured~\cite{talvitie2018learning, pan2021effects}. 
However, misspecification of rewards has not been examined from the perspective of being carried out by an adversary. 
An intelligent, AI-enabled adversary might have an ability to modify the reward function, which will distract the system from learning behaviors to accomplish a specified goal. 
We will design methods to effectively identify distracting or deceptive rewards, and develop principled solutions to minimize the impact of deceptive rewards. 
Our solutions will use insights from distributed consensus to determine the fraction of agents that are contributing to a distracting reward and establish bounds on the maximum number of distracting rewards that can be handled, and game theory to learn best-response behaviors for the system despite the presence of such rewards.

\noindent {\bf Leveraging Reward Function Structure}:
When information about the environment is available, potential functions have been used to shape the rewards provided by the environment in order to accelerate the learning process~\cite{ng1999policy, devlin2011empirical, devlin2014potential, harutyunyan2015expressing}. 
However, using such structured rewards when domain knowledge is not available remains a challenge. 
Initial research in this direction has examined the use of graph neural networks~\cite{klissarov2020reward} and hindsight~\cite{mesnard2021counterfactual}. 
We will develop generalized hierarchical solutions that will first learn an underlying structure of a reward function, and then learn policies that will maximize a reward function consistent with the learned structure. 
Once a reward structure has been learned, an interesting question is whether the structure will help identifying adversarial interactions. 
Such interactions will likely contribute to modifying rewards in a manner that will violate the structural properties learned. 
We will design algorithms that will use structural to identify and reduce the impact of adversarial interactions.

Selecting a policy that maximizes a reward function is computationally challenging when the action space is high-dimensional and/or combinatorial in nature. 
We will investigate and develop algorithms that exploit structure in order to compute optimal policies as well as accelerate the learning process. 
One structure that we have exploited in our prior work is submodularity, which is a diminishing-returns property of discrete functions. 
In~\cite{sahabandu2021scalable}, we proved that near-optimal policies can be computed in polynomial time for Markov decision processes that possess an \emph{approximate transition independence} property and have submodular rewards. 
This preliminary approach, however, assumed that the transition matrix of the MDP was known a priori, leaving the generalization to multi-agent RL scenarios as a problem for future work. 
In cases where the reward function is not submodular, we will explore properties such as curvature~\cite{sviridenko2017optimal} and weak submodularity~\cite{liu2018controlled, liu2018minimal} to construct scalable algorithms with application-dependent optimality bounds.



\noindent \textbf{3. Reinforcement Learning}

We will present some additional and more advanced research directions along the line of reinforcement learning. 

\noindent {\bf Deep Reinforcement Learning.}
Our proposal falls within the larger study of deep reinforcement learning, which aims to learn RL agents with policies via deep neural networks.
Reinforcement learning (RL) is the branch of machine learning that addresses automated decision making and control.
While most machine learning methods deal with problems of prediction and recognition, RL specifically addresses decision making under uncertainty to maximize a utility function.
This highly general formulation encompasses a broad range of potential AI applications in the real world: from controlling autonomous vehicles and robots to regulating power grids to managing inventory levels in warehouses to optimizing medical treatment trials.
The real world is full of complex decision making problems that require handling rich inputs and making rational decisions in stochastic systems under uncertainty.

Deep RL has been successfully applied to learning complex control policies, including video games~\cite{mnih2015human,openai2018dota,arulkumaran2017deep}, classical games like Go~\cite{silver2017alphago}, simulated~\cite{schulman2015trust,lillicrap2015continuous} and real~\cite{levine2016end,kalashnikov2018qt,openai2019solving,nagabandi2019deep} robot control.

This paradigm has also been extended to related setups in learning from demonstrations, such as imitation learning~\cite{ho2016generative,finn2016connection,kuefler2017imitating,fu2017learning}, off-line reinforcement learning~\cite{fujimoto2019off,kumar2019stabilizing,kumar2020conservative,kidambi2020morel,levine2020offline,yu2020mopo}, and observation-only demonstrations~\cite{liu2017imitation,torabi2018generative}.

\noindent {\bf Unsupervised Reinforcement Learning.}
More recently, there has been a body of research in \textit{unsupervised} reinforcement learning where agents learn from reward-free interactions by optimizing for self-supervised intrinsic reward proxies for exploration and skill discovery~\cite{pathakICMl17curiosity,pathak2019self,eysenbach2018diversity,hansen2019fast, sharma2019dynamics,liu2021aps,liu2021behavior,yarats2021reinforcement}.
While currently largely contained to pre-training for standard RL environments (games, simulated robotic control), unsupervised RL has a natural fit with security and program verification: indeed, an unsupervised RL agent is effectively tasked with visiting all possible states a program or machine can end up in, i.e. achieve maximal coverage, which (with further investigation) could lead to a new generation of debugging and security tools.

\noindent {\bf Multi-Agent Reinforcement Learning.}
While the most standard reinforcement lerning setting considers one agent maximizing its reward, many practical applications involve multiple agents that need to collaborate or might need to compete.
Methods like Multi-Agent Actor-Critic~\cite{lowe2017multiagent} allow for effective learning in such multi-agent settings thanks to variance reduction techniques.
It has been shown that multi-agent environments can lead to the emergence of communication~\cite{foerster2016learning, sukhbaatar2016learning, lazaridou2016multi, mordatch2017emergence}---as the emergence of communication enables the agents to more effectively optimize their rewards.

Multi-agent learning has also been a powerful tool to aid exploration for single-agent RL.
Having agents compete leads to a natural curriculum, where initially the environment is easier and as the agents learn the environment becomes gradually more difficult~\cite{sukhbaatar2017intrinsic,florensa2018automatic,racaniere2019automated,openai2021asymmetric,dennis2020emergent}.


PI Abbeel is one of the leading researchers in Reinforcement Learning, having pioneered several of the main breakthroughs, including the first stable on-policy Deep RL algorithm Trust Region Policy Optimization (TRPO) \cite{schulman2015trust}, the current SOTA off-policy actor critic algorithm Soft-Actor Critic (SAC) \cite{haarnoja2018soft}, efficient multi-agent actor-critic~\cite{lowe2017multiagent}, emerging communication~\cite{mordatch2017emergence}, representation learning in RL (CURL and RAD) \cite{misha20curl,RAD}, unsupervised RL algorithms and benchmark \cite{URLB}, RL from human preferences algorithms (PEBBLE) \cite{pebble} and benchmark (B-PREF) \cite{lee2021bpref}.
Each of these contributions was accompanied by open-source releases, and have seen significant pick-up by the community.
The Abbeel lab also released the first open-source Deep RL library rllab \cite{duan2016benchmarking}.
This library directly resulted in the OpenAI Gym interface \cite{openaigym} which is still the prevalent environment interface today.


\noindent \textbf{4. Multi-agent network as moving target}

\textcolor{red}{xxx- Need contents on this topic. Or we may remove this part, as it may have been covered in a different place(s). }



% E. References Cited
\newpage
\renewcommand\refname{References Cited}
\bibliography{gao,Abbeel,Distributed-learning,reference}
% I prefer to use the IEEE bibliography style.
% That's  NOT required by the NSF guidelines.
% Feel Free to use whatever style you prefer
\bibliographystyle{IEEEtran}

\end{document}
